The Internet has an immense importance in our day to day life, but at the same time, it has become the most advantageous medium of infecting computers, attacking users, and distributing malicious code. As JavaScript is the principal language of client side programming, it is frequently used in conducting such attacks. Various approaches have been made to overcome the JavaScript security issues. Some advanced approaches utilize machine learning technology in combination with deobfuscation and emulation. Many methods of analysis incorporate static analysis and dynamic analysis. Our solution is entirely based on static analysis, which avoids unnecessary runtime overhead.

The primary objective of this project is to integrate the previous work on Towards A Robust Detection of Malicious JavaScript (TARDIS) into the web browser via a Firefox add-on and to demonstrate the usability of our add-on in defending against such attacks. TARDIS combines statistical language modeling for automatic feature extraction with structural features from an abstract syntax tree. We have developed a Firefox add-on that is capable of extracting JavaScript code from the page visited and classifying the JavaScript code as either malicious or benign. We leverage the benefit of using a pre-compiled training model in JavaScript Object Notation (JSON). JSON is lightweight and does not consume much memory on a user's machine. Moreover, it stores the data as key-value pairs and easily maps to the data structures used in modern programming languages. The principle advantage of using a pre-compiled training model is better performance. Our model can achieve 98\% accuracy on our sample dataset.






